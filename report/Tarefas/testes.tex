
\section{Testes}

Os testes do projeto foram organizados em pastas de testes específicos localizados em \textit{/test} 
no projeto \textit{RCP-MO601-Project-01}. Cada pasta de teste contém a especificação do circuito 
lógico desejado (\textit{circuito.hdl}), os estímulos (\textit{estimulos.txt}) a serem processados ao 
longo do tempo (valores de variáveis e/ou indicadores de tempo), 
os arquivos com as saídas esperadas (\textit{esperado0.csv e esperado1.csv}) e os arquivos resultantes da simulação 
(\textit{saida0.csv e saida1.csv}).

Novos testes podem ser adicionados à pasta principal de testes (\textit{/test}), contudo o simulador exige a
existência dos arquivos \textit{circuito.hdl} e \textit{estimulos.txt}. Além disso, não existe limite para a quantidade 
de testes incluídos, sendo que o simulador processará todos aqueles que existirem na pasta principal de testes.

Durante a simulação dos testes, conforme pode-se observar no \textit{algorithm \ref{alg:simulacao}}, 
todas as pastas de testes específicos são consideradas, seus arquivos de entrada são 
processados e os resultados (saídas) são armazenados nas mesmas.